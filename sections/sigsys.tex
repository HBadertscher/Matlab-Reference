\subsection{Signale}

\begin{tabular}{ll}
	\texttt{p=polyfit(x,y,n)} & liefert n+1 Polynomkoeffizienten der nichtlinearen Kennlinie (Reihenfolge: $c_n, c_{n-1}, ..., c_0)$ \\
	\texttt{f=polyval(p,x)} & liefert Polynomwerte an den Messpunkten x \\
\end{tabular}


\subsection{Systeme}

\subsubsection{Modellierung}
\begin{tabular}{|l|l|}  \hline
	\multirow{3}{*}{Simulink}
		 	& Modell in Simulink erstellen \\
			& \texttt{[a b c d] = linmod2('SimulinkModell');} \\
			& \texttt{sys = ss(a,b,c,d);} \\ \hline
	\multirow{4}{*}{Transferfunktion}
			& für Polynom $H(s) = \frac{a_n s^n + a_{n-1} s^{n-1} + ... + a_0}{b_m s^m + b_{m-1} s^{m-1} + ... + b_0}$: \\
			& \texttt{sys = tf([$a_n$ $a_{n-1}$ ... $a_0$],[$b_m$ $b_{m-1}$ ... $b_0$])} \\ \cline{2-2}
			& für Polynom $H(s) = \frac{(s-n_0) (s-n_1) ... (s-n_n)}{(s-p_0) (s-p_1) ... (s-p_m)}$: \\
			& \texttt{sys = tf(poly([$n_0$,$n_1$, ... , $n_n$],poly([$p_0$,$p_1$, ... ,$p_m$])))}	\\ \hline
\end{tabular} 

\subsubsection{Funktionen}
\begin{tabular}{ll}
	\texttt{bode(sys)} & Bode Diagramm des Systems \\
	\texttt{impulse(sys)} & Impulsantwort des Systems \\
	\texttt{step(H)} & Schrittantwort des Systems \\
	\texttt{pzmap(sys)} & Pol- / Nullstellendiagramm des Systems \\
	\texttt{nicholsplot(sys)} & Nichols-Diagramm des Systems \\
	\texttt{margin(sys)} &  Bode Diagramm mit Amplituden- und Phasenrand \\
	\texttt{allmargin(sys)} & Berechnet Amplituden- und Phasenrand \\
	\texttt{ltiview(sys)} & Darstellung verschiedener Diagramme (Bode, Nichols, ...) \\
	\texttt{sisotool(sys)} & GUI zur Stabilisierung des Systems (Feedback, ...) \\
	\texttt{minreal(sys)} & Gibt gekürzte UTF zurück \\
\end{tabular}


\subsubsection{Zusammenschalten von Systemen (SFD)}
\begin{minipage}{12cm}
\textbf{High-Level Funktionen:} \\
\begin{tabular}{|l|l|} \hline
	\multirow{2}{*}{Serieschaltung (*)}
			& \texttt{serie\_sys = sys1 * sys2} \\
			& \texttt{serie\_sys = series(sys1,sys2)} \\ \hline
	\multirow{2}{*}{Parallelschaltung (+/-)}
			& \texttt{parallel\_sys = sys1 + sys2} \\
			& \texttt{parallel\_sys = parallel(sys1, sys2)} \\ \hline
	Feedback & \texttt{feedback\_sys = feedback(sys1,sys2,+1)} \\ \hline
\end{tabular}
\end{minipage}
\begin{minipage}{8cm}
\textbf{Append / Connect:} \\
	\texttt{sys = append(sys1,sys2,...)} \\
	\texttt{inputs = [a]} \textit{\% Eingang von System a} \\
	\texttt{outputs = [b]} \textit{\% Ausgang von System b} \\
	\texttt{Q = [...]}  \textit{\% Matrix: Beschaltung der Systeme} \\
	\texttt{sysc = connect(sys,Q,inputs,outputs)} \\
\end{minipage}

