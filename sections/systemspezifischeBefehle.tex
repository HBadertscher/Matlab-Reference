\begin{tabularx}{\textwidth}{X|X}
  \begin{tabular}{ll}
      \multicolumn{2}{l}{\textbf{Befehle}}
    \\
      \texttt{clc}      & Commandfenster löschen
    \\
  		\texttt{clear}    &	Alle Variabelnzuweisungen löschen
  	\\
  	  \texttt{help}     & Hilfe zu einer Funktion
  	\\
      \texttt{load}     & Variabelnzuweisung wieder laden
    \\
      \texttt{lookfor}  & Befehle nach einem Suchbegriff durchsuchen
    \\
      \texttt{save}     & Alle Variabelnzuweisungen speichern
    \\
      \texttt{what}     & Dateien im aktuellen Verzeichnis anzeigen
    \\
      \texttt{whos}     & Verwendete Variabeln anzeigen
    \\
      \texttt{;}        & Befehl ausführen ohne anzuzeigen
    \\
      \texttt{simplify( )} & Ausdruck vereinfachen
    \\
      \texttt{pretty( )}   & Ausdruck in lesbarer Form anzeigen
    \\
      \texttt{grid on}  & Gitterlinien in einem Diagramm anzeigen
  \end{tabular}
  
& %--------------------------------------------------------------

  \begin{tabular}{ll}
    \multicolumn{2}{l}{\textbf{Operatoren}}
    \\
      \texttt{*}        & Matrix Multiplikation
    \\
      \texttt{.*}       & Array Multiplikation
    \\
      \texttt{./}       & Array Division
    \\
      \texttt{\hoch}    & Matrix Exponent
    \\
      \texttt{.\hoch}   & Array Exponene
    \\
      \texttt{'}        & Transponiert (mit konjugiert komplexen Zahlen)
    \\
      \texttt{.'}       & Transponiert (ohne Veränderung der komplexen Zahlen)
    \\
      \texttt{\textbackslash} & Lösung eines linearen Gleichungssystems
  \end{tabular}
  
\\
  \begin{tabular}{ll}
    \multicolumn{2}{l}{\textbf{Variabeln}}
    \\
      \texttt{ans}      & Letztes Resultat
    \\
      \texttt{pi}       & $\pi$
    \\
      \texttt{i, j}     & Imaginäre Einheit
  \end{tabular}
  
  & %---------------------------------------------------------------------
  \begin{tabular}{ll}
    \multicolumn{2}{l}{\textbf{Diverse}}
    \\
      \texttt{rand(x)}     & Zufallszahlen zwischen 0 und 1 gleichverteilt
    \\
      \texttt{randn(x)}    & Zufallszahlen um 0 mit Gaussverteilung
  \end{tabular}
  
\end{tabularx}